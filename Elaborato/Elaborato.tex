\documentclass[12pt, twoside, letterpaper]{article}
\usepackage[top=2cm,bottom=4cm,left=3cm,right=3cm,asymmetric]{geometry}
\usepackage{color}   %May be necessary if you want to color links
\usepackage{hyperref}
\usepackage[utf8x]{inputenc}
%\usepackage[table]{xcolor}
\usepackage[english]{babel}
\usepackage{amsmath, amsthm, amssymb, amsfonts}
\usepackage[breakable]{tcolorbox}
\usepackage{floatrow}
\usepackage{graphicx}
\graphicspath{ {./img/} }
\usepackage{titling}
\usepackage{caption}
\usepackage[nouppercase]{frontespizio}
\newtcolorbox{dati}[1][]{colback=green!30!white, bottomtitle=1.5mm,breakable,#1}
\newtcolorbox{variabili}[1][]{colback=red!30!white, bottomtitle=1.5mm,breakable,#1}
\newtcolorbox{vincoli}[1][]{colback=orange!30!white, bottomtitle=1.5mm,breakable,#1}
\newtcolorbox{obiettivo}[1][]{colback=yellow!30!white, bottomtitle=1.5mm,breakable,#1}
\newenvironment{rcases}
  {\left.\begin{aligned}}
  {\end{aligned}\right\rbrace}

\pretitle{
	\begin{center}
	\LARGE
	\includegraphics{tesiSCIENZE_TECNOLOGIE.jpg}\\
	Corso di laurea in informatica\\
	
}
\posttitle{\end{center}}
\title{Uso di reti neurali per la classificazione di dati in problemi di medicina legale}
\author{Mario Petruccelli \cr Università degli studi di Milano}
\date{A.A. 2019/2020}

\addto\captionsenglish{% Replace "english" with the language you use
  \renewcommand{\contentsname}%
    {Sommario}%
}

\begin{document}

	\begin{titlepage}
		\maketitle
		\newpage
		\tableofcontents
	\end{titlepage}


	\section{Reti neurali}
		\subsection{Che cos'è una rete neurale?}
		\subsection{Apprendimento supervisionato}
		\subsection{Componenti principali di una rete neurale}
		\subsection{Rete neurale feedforward.}
		\subsection{Discesa del gradiente e back propagation.}
	\section{Tecniche utilizzate}
		\subsection{Cross validation}
		\subsection{Grid search CV}
		\subsection{Standard Scaler}
		\subsection{PCA}
		\subsection{TSNE}
		\subsection{SMOTE}
	\section{Esperimenti fatti}
		\subsection{Iris dataset}
		\subsection{Dataset incidenti}
		\subsection{Vari esperimenti fatti sul dataset degli incidenti}
		\subsection{Tecniche usate per aumentare i dati}

				
\end{document}



