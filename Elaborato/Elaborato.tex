\documentclass[12pt, twoside, letterpaper]{report}
\usepackage[top=2cm,bottom=4cm,left=3cm,right=3cm,asymmetric]{geometry}

\usepackage{fancyhdr}
\pagestyle{fancy}
\fancyfoot[C]{}    
\fancyfoot[LE,RO]{\thepage}        
\fancyhead[RO]{\slshape \rightmark}
\fancyhead[LE]{\slshape\leftmark}      
\fancyhead[RE,LO]{}  

\usepackage{color}   
\usepackage{hyperref}
\usepackage[utf8x]{inputenc}
\usepackage[italian]{babel}
\usepackage{amsmath, amsthm, amssymb, amsfonts}
\usepackage{blindtext}
\usepackage{cite}
\usepackage[breakable]{tcolorbox}
\usepackage{pdfpages}
\usepackage{enumitem}
\usepackage{caption}
\usepackage{subcaption}
\usepackage{graphicx}
\graphicspath{ {./img/} }
\newcommand{\img}[4] {
	\begin{figure}[h]
		\caption{#1}
		\centering
		\includegraphics[scale=#2]{#3}\\
		\label{#4}
	\end{figure}
}

\title{Uso di reti neurali per la classificazione di dati in problemi di medicina legale}
\author{Mario Petruccelli \cr Università degli studi di Milano}
\date{A.A. 2019/2020}

\begin{document}

	\begin{titlepage}
		\includepdf{img/FRONTESPIZIO.pdf}
		\newpage
		\tableofcontents
	\end{titlepage}

	\chapter*{Introduzione} \markboth{Introduzione}{}
		Qua ci andrà l'introduzione

		\newpage		
	\chapter{Reti neurali}
		Le \textbf{reti neurali artificiali} sono un campo molto importante del \textit{machine learning}. Consistono in un modello matematico che prende ispirazione dalle reti neurali biologiche presenti nel cervello animale. In questo capitolo ne vedremo le loro componenti e il loro funzionamento. Ma cos'è il machine learning?

		\section{Paradigma del machine learning}
			Il machine learning è una disciplina scientifica dell'ambito dell'intelligenza artificiale ed è una di quelle in più rapida espansione; negli ultimi decenni è diventato di uso comune in moltissime applicazioni che richiedono l'estrazione di informazioni dai dati. Lo scopo di tale disciplina è quello di individuare dei pattern significativi nei dati, che difficilmente sarebbero riconoscibili con l'uso tradizionale della programmazione \textit{(e.g. riconoscere la razza di un cane partendo da una foto)}.  
			
			Prendendo come esempio noi umani, molte delle nostre abilità vengono acquisite o raffinate dalla nostra esperienza, e così \textit{apprendiamo}. Immaginiamo di trovarci davanti un essere vivente che non abbiamo mai visto prima. Nonostante questo essere sia nuovo ai nostri occhi, sapremo riconoscere con molta probabilità se questo si tratta di un insetto, di un mammifero o di una pianta. Lo stesso concetto possiamo provare ad applicarlo alle macchine, ma come possono apprendere? 
			
			\subsection{Tipi di apprendimento} L'apprendimento è un dominio di ampio spettro e di conseguenza, il machine learning è diviso in diversi sottocampi in cui abbiamo più paradigmi di apprendimento. La principale distinzione che possiamo fare è la differenza tra l'apprendimento \textbf{supervisionato} e \textbf{non supervisionato}.
			
				\paragraph{Apprendimento supervisionato} Consideriamo di dover classificare tre specie di fiori leggermente diversi tra loro, appartenenti alla stessa famiglia \textit{(esperimento che riprenderemo più avanti)}. Per imparare a riconoscerli, potremmo ricevere un insieme di dati \textit{(e.g. larghezza dei petali, lunghezza del gambo, ecc...)} appartenenti ai fiori di cui conosciamo già l'etichetta. Dopo questa fase di \textbf{allenamento}, dovremmo essere in grado di trovare un pattern tra i dati per etichettare nuovi fiori, questa volta privi di etichetta e non appartenenti all'insieme dato in precedenza. 
				
					In maniera più astratta, possiamo vedere questo come un processo di \textit{utilizzo dell'esperienza passata per acquisire competenza}. L'apprendimento supervisionato descrive uno scenario in cui l'esperienza, nel nostro caso un insieme di \textbf{training}, contiene delle etichette che non sono presenti nell'insieme di \textbf{test}, a cui dobbiamo applicare l'esperienza acquisita in modo da \textbf{predire} l'informazione mancante.
				
				\paragraph{Apprendimento non supervisionato}  Nell'apprendimento non supervisionato non c'è distinzione tra i dati di training e di test. Vengono processati i dati con lo scopo di tirare fuori una versione compressa, o un "riassunto" di essi. Il raggruppamento dei dati in sottoinsiemi con caratteristiche simili è una delle pratiche più usate in questo campo.\\\\
				Il modello di cui ci occuperemo noi, ovvero le reti neurali artificiali, utilizzano un'apprendimento di tipo supervisionato. Andiamo a vederle più nel dettaglio.
			
		\section{Che cos'è una rete neurale?}
			Come abbiamo già anticipato, le reti neurali artificial	i cercano di riprendere quello che sono le loro omonime biologiche. Il cervello animale è un meccanismo in grado di apprendere grazie all'esperienza, ed è per questo che si è cercato di emularlo \textit{(in modo molto semplificato)} come modello di machine learning. Probabilmente saprete già che il cervello è composto da \textbf{neuroni}, ed essi sono connessi tra di loro da delle \textbf{sinapsi} in modo da poter comunicare. Ebbene, anche il nostro modello artificiale riprende questa struttura.
			
		\subsection{Componenti principali di una rete neurale} 
			\paragraph{Concetto di tempo} Prima di dare una definizione, introduciamo il concetto di tempo in una rete neurale. Ci si riferisce al tempo vigente come $(t)$, al passo successivo come $(t+1)$, al passo precedente come $(t-1)$, e così via.
			\paragraph{Rete neurale artificiale}
			Una rete neurale artificiale può essere vista come un grafo, cioè consiste in un insieme di \textbf{nodi} \textit{(neuroni)} e \textbf{archi orientati pesati} \textit{(sinapsi)}. Più formalmente, una rete neurale è una tripla $(N,V, \omega)$ in cui:
			\begin{itemize}
				\item $N$ è l'insieme dei neuroni.
				\item $V = \{(i,j) | i,j \in N\}$ è l'insieme delle connessioni tra i neuroni $i$ e $j$.
				\item $\omega: V \rightarrow \mathbb{R}$ è la funzione di peso, dove $\omega_{i,j}$ è il peso della connessione tra i neuroni $i$ e $j$.
			\end{itemize}
			Un neurone, per essere definito tale deve essere in grado di elaborare i dati che gli arrivano tramite le sue connessioni. Questo accade attraverso tre funzioni: funzione di \textbf{propagazione}, funzione di \textbf{attivazione} e funzione di \textbf{uscita}. 
			
			\img{Struttura di un neurone}{0.4}{neurone.png}{neurone}
			
			 \paragraph{Funzione di propagazione} Dato un neurone $j$, la sua funzione di propagazione riceve gli output $o_{i_1}, \dots, o_{i_n}$ dei neuroni $i_1, \dots, i_n$ connessi a $j$ e i corrispettivi pesi $\omega_{i,j}$, restituendo un valore detto \textit{network input} net$_j$ che verrà poi processato dalla \textit{funzione di attivazione}. 
			 
			 	Sia $I = \{i_1, i_2, \dots, i_n\}$ l'insieme dei neuroni tale che $\forall z \in \{1, \dots, n\} : \exists w_{i_z,j}$. Allora il network input di $j$ ($net_j$), è calcolato dalla funzione di propagazione $f_{prop}$ come segue: $$net_j = f_{prop}(o_{i_1}, \dots, o_{i_n},w_{i_1,j}, \dots, w_{i_n,j})$$
			 	La funzione di propagazione più usata è la \textbf{somma pesata}, ovvero la somma delle moltiplicazioni dell'output di ogni neurone $i$ per il corrispettivo peso $\omega_{i,j}$: $$net_j = \sum_{i \in I} o_i w_{i,j}$$
			 	
			 \paragraph{Stato di attivazione e valore di soglia} Così come in natura, un neurone può essere \textit{attivo} o meno. Lo \textbf{stato di attivazione} \textit{(o attivazione)} è una reazione ai valori che abbiamo ricevuto in input, e il neurone viene attivato quando il \textit{network input} supera il \textbf{valore di soglia} di quel neurone. Sia $j$ un neurone. Il \textbf{valore di soglia} $\Theta_j$ è assegnato unicamente a $j$ e indica la posizione del valore massimo del gradiente della funzione di attivazione.
			 
			 \paragraph{Funzione di attivazione} Sia $j$ un neurone. La funzione di attivazione è così definita: $$a_j(t) = f_{act}(net_j(t),a_j(t-1), \Theta_j)$$
			 	Trasforma il \textit{network input} net$_j$, il precedente stato di attivazione $a_j(t-1)$, e il valore di soglia $\Theta_j$ in un nuovo stato di attivazione $a_j(t)$ 
			 	A differenza di molte delle variabili definite prima, la funzione di attivazione viene definita \textit{globalmente} per tutti i neuroni. Esistono diverse funzioni di attivazione, alcune tra le più usate, come possiamo vedere in figura~\ref{fig:funzioni di attivazione} sono: 
			 	\begin{itemize}
			 		\item La funzione logistica: $f(x) = \frac{1}{1+e^{-x}}$
			 		\item La funzione tangente iperbolica: $f(x) = tanh(x)$
			 		\item La funzione ReLU: $f(x) = max(0,x)$
			 		\item La funzione identità: $f(x) = x$
			 	\end{itemize}
			 	
			 \begin{figure}[h]
				\begin{subfigure}[]{.5\textwidth}
					\centering
					\includegraphics[width=.9\linewidth]{logistic.png}
					\caption{Funzione logistica}
					\label{fig:logistic}
				\end{subfigure}
				\hfill
				\begin{subfigure}[]{.5\textwidth}
					\centering
					\includegraphics[width=.9\linewidth]{tanh.png}
					\caption{Funzione tangente iperbolica}
					\label{fig:tanh}
				\end{subfigure}
				
				\bigskip  
				
				\begin{subfigure}[]{.5\textwidth}
					\centering
					\includegraphics[width=.9\linewidth]{relu.png}
					\caption{Funzione ReLU}
					\label{fig:relu}
				\end{subfigure}
				\hfill
				\begin{subfigure}[]{.5\textwidth}
					\centering
					\includegraphics[width=.9\linewidth]{identity.png}
					\caption{Funzione di identità}
					\label{fig:identity}
				\end{subfigure}
				
				\caption{Funzioni di attivazione}
				\label{fig:funzioni di attivazione}
			\end{figure}	
			 	
			 \paragraph{Funzione di uscita} La funzione di uscita di un neurone $j$ calcola i valori che verranno trasmessi agli altri neuroni connessi a $j$. 
			 
			 	Sia $j$ un neurone. La funzione di uscita calcola il valore di output $o_j$ del neurone $j$ in funzione dell'attivazione $a_j$. $$f_{out}(a_j) = o_j$$  
			 	
			 	Anche in questo caso, solitamente la funzione è definita globalmente. Spesso viene usata come funzione di uscita la \textit{funzione di identità}, mandando in output direttamente l'attivazione $a_j$. $$f_{out}(a_j) = a_j, \text{ quindi } o_j = a_j$$
			 	
			 \paragraph{Neurone bias} Come abbiamo visto, il valore di soglia (che ci dice quando un neurone viene considerato attivo) è un parametro della funzione di attivazione. Siccome sarebbe complicato accedere alla funzione di attivazione a \textit{runtime} per modificare dato valore, i valori di soglia $\Theta_{j_1}, \dots, \Theta_{j_n}$ dei neuroni $j_1, \dots, j_n$ possono essere realizzati come pesi di un neurone sempre attivo. Aggiungiamo quindi un neurone \textit{bias} il cui \textit{output} è fisso a 1, connesso ai neuroni $j_1, \dots, j_n$ il cui peso delle connessioni è pari a $-\Theta_{j_1}, \dots, -\Theta_{j_n}$
			 
			 	\img{Due reti equivalenti, a destra con neurone bias e a sinistra senza.}{0.5}{bias-neuron.png}{bias}
			 	
			 \paragraph{Vettore d'ingresso e vettore di uscita} Le reti neurali che andremo a vedere \textit{(i.e. le reti neurali feed-forward)}, fanno parte di quella categoria di reti che processano dei dati in input, per poi produrre un output. Una rete con $n$ neuroni di input, necessita di un \textbf{vettore d'ingresso} $x = (x_1, x_2, \dots, x_n)$ che gli verrà dato in pasto, e dati $m$ neuroni di uscita fornisce un \textbf{vettore di uscita} $y = (y_1, y_2, \dots, y_m)$. 
			 
			 	 			 
		\section{Reti neurali feed-forward}
			Come possiamo immaginare, esistono diverse topologie di rete, ma noi ci concentreremo sulle \textbf{reti neurali feed-forward} \textit{(o in italiano, reti neurali con flusso in avanti)}. In queste reti, i neuroni sono raggruppati nei seguenti \textbf{strati}: \textit{uno strato di ingresso, $n$ strati nascosti} e \textit{uno strato di uscita}. In una rete feed-forward ogni neurone ha connessioni dirette solo con lo strato successivo al suo (in direzione dello strato di uscita), evitando così l'esistenza di cicli. Una rete in cui ogni neurone è connesso a tutti i neuroni dello strato successivo, viene detta rete \textbf{completamente connessa}.
			
			\paragraph{Percettrone} Un percettrone è una rete feed-forward completamente connessa costituita da uno strato d'ingresso in cui i neuroni di input non fanno altro che propagare l'informazione ricevuta \textit{(neuroni identità)}, seguito  da almeno uno strato di pesi addestrabili.
			
			\paragraph{Percettrone a singolo strato} Un percettrone a singolo strato è la rete feed-forward più semplice che si possa fare. Esso è costituito semplicemente da uno strato d'ingresso e uno di uscita. 
				\img{Percettrone a singolo strato con 5 neuroni di ingresso e 3 neuroni di uscita}{0.5}{slp.png}{slp}
				
			\paragraph{Percettrone multistrato} Un percettrone multistrato è una rete feed-forward che ha più di uno strato di pesi addestrabili. Si può dimostrare che i percettroni a singolo strato siano in grado di rappresentare solo dati separabili \textbf{linearmente}, mentre i percettroni multistrato ci permettono di aggirare questa limitazione. 
				\img{percettrone multistrato}{0.5}{nn-feed-forward.png}{feedforward}
		
		\section{Apprendimento di una rete neurale}
			Abbiamo visto la struttura delle reti neurali ma ancora non sappiamo niente su come esse apprendano. Sappiamo che in una rete feed-forward ogni neurone è collegato tramite degli archi pesati a tutti i neuroni dello strato successivo, ma non sappiamo come i pesi vengano inizializzati e in base a cosa vengano modificati. In realtà il processo di apprendimento può essere descritto in pochi passi: 
			\begin{itemize}
				\item Dare in pasto alla nostra rete un vettore d'ingresso $p$ di cui sappiamo l'output desiderato.
				\item Propagare in avanti l'input attraverso le tre funzioni viste prima.
				\item Comparare il vettore di uscita $y$ ottenuto con il vettore di uscita desiderato $t$: sottraendo i due vettori otteniamo un \textbf{vettore di errore} $E_p = t - y$.
				\item Correggere i pesi con lo scopo di ridurre al minimo la differenza $t-y$. 
			\end{itemize}
			Questo processo, nonostante la differenza del vettore di errore sia minima, non ci dice esattamente se la nostra rete abbia appreso il pattern che c'è dietro i dati, o se ha semplicemente imparato a riprodurre l'output dei dati che gli abbiamo dato in ingresso. Questo fenomeno è chiamato \textbf{overfitting}, e per accorgerci di questo problema è una buona prassi dividere in due insiemi i dati che usiamo per allenare la rete neurale in: 
			\begin{itemize}
				\item Un insieme di \textbf{training} utilizzato per allenare la rete.
				\item Un insieme di \textbf{validation} utilizzato per valutare la bontà della rete.
			\end{itemize}
			La divisione deve essere casuale in modo tale che i dati siano influenzati il meno possibile da fattori esterni (e.g. prendere i primi dati in un dataset in cui essi sono ordinati in ordine crescente potrebbe influire negativamente l'apprendimento) e solitamente si tiene una proporzione in cui l'insieme di training ha un numero di dati maggiore rispetto all'insieme di validation (almeno 3/4 dei dati totali). 
			
			\paragraph{Apprendimento online e offline} Oltre alla principale distinzione vista in precedenza tra i tipi di apprendimento (supervisionato e non supervisionato), c'è un'altra distinzione su come ci approcciamo alla modifica dei pesi. L'apprendimento può essere: 
				\begin{itemize}
					\item \textbf{Online:} i dati di training vengono inseriti uno a uno, la rete calcola l'errore del singolo dato e modifica i pesi di conseguenza.
					\item \textbf{Offline:} diversi dati di training vengono inseriti nella rete insieme, la rete accumula gli errori e modifica i pesi di conseguenza. Ogni lotto di dati viene detto \textbf{batch}, e tutta la procedura di apprendimento di un batch viene detta \textbf{epoca}.
				\end{itemize}
				
			\subsection{Curva di apprendimento}
				La curva di apprendimento traccia il progresso dell'errore e ci aiuta a capire se la nostra rete stia facendo progressi o meno. Per questo scopo dobbiamo normalizzare l'errore. Il modo più comune per fare ciò è quello di calcolare il valore efficace tra il vettore di uscita desiderato $t$ e il vettore di uscita $y$.  

				Siano $\Omega$ i neuroni di uscita e $O$ l'insieme dei neuroni di uscita. Il valore efficace si calcola così: $$Err_p = \sqrt{\frac{\sum_{\Omega \in O} (t_{\Omega} - y_{\Omega})^2}{|O|}}$$
				Per quanto riguarda l'apprendimento offline, l'\textbf{errore totale} di un'epoca si calcola nel seguente modo: $$Err = \sum_{p \in P} Err_p$$
				La curva di apprendimento va tracciata per i due insiemi di training e validation, in questo modo possiamo accorgerci se siamo in un caso di overfitting guardando solo l'andamento delle due curve: se la curva di training decresce velocemente mentre quella di validation comincia a risalire siamo chiaramente in un caso in cui stiamo perdendo la generalizzazione dei dati che avevamo acquisito, l'ideale sarebbe fermarsi appena prima che la curva di validation cominci a risalire (\textbf{early stopping}). Solitamente la curva di training avrà comunque un errore minore rispetto quella di validation.

				\begin{figure}[h]
				\begin{subfigure}[]{.5\textwidth}
					\centering
					\includegraphics[width=\linewidth]{learning_curve.png}
					\caption{Curva di apprendimento regolare}
					\label{fig:learning_curve}
				\end{subfigure}
				\hfill
				\begin{subfigure}[]{.5\textwidth}
					\centering
					\includegraphics[width=\linewidth]{overfitting.png}
					\caption{Corva di apprendimento con overfitting}
					\label{fig:overfitting}
				\end{subfigure}
				
				\caption{Curve di apprendimento degli insiemi di training e validation}
				\label{fig:curve_di_apprendimento}
				\end{figure}
				
			
			\subsection{Discesa del gradiente}
				Ora ci dovrebbe venire spontaneo chiederci: quando smettiamo di apprendere? Non è una domanda dalla risposta facile, e per capire questo fondamentale passaggio bisogna avere un'idea di cosa sia la discesa del gradiente. È una procedura che viene utilizzata per massimizzare o minimizzare una funzione. Il gradiente è un vettore $g$ definito in ogni punto derivabile di una funzione ed esso punta verso la salita più ripida. Di conseguenza $-g$ punterà verso la discesa più ripida.
				\img{Discesa del gradiente di una funzione a due dimensioni}{0.5}{gradient_descent_2d.png}{gradient_descent}

				\paragraph{Gradiente} Sia $g$ un gradiente. Allora $g$ è il vettore con $n$ componenti che è definito in ogni punto derivabile di una funzione a $n$ dimensioni $f(x_1, x_2, \dots, x_n)$. L'operatore gradiente è definito come $$g(x_1, x_2, \dots, x_n) = \triangledown f(x_1, x_2, \dots, x_n)$$
					$g$ punta da ogni punto derivabile di $f$ verso la salita più ripida, con una pendenza pari a $|g|$. \cite{kriesel}
				
				\paragraph{Discesa del gradiente} Sia $f$ una funzione a $n$ dimensioni e $s=(s_1, s_2, \dots, s_n)$ il dato punto di partenza. \textbf{Discesa del gradiente} significa andare da $f(s)$ in direzione di $-g$, con passi della grandezza di $|g|$ verso valori sempre minori di $f$. \cite{kriesel}
				
				\img{Problemi della tecnica della discesa del gradiente}{0.4}{gradient_descent.png}{gradient_descent}
				
				Questa tecnica purtroppo non è priva di problemi, potremmo cadere in diverse "trappole" senza accorgercene. Alcuni esempi sono: 
				\begin{itemize}
				 	\item Incagliarsi su un minimo locale non soddisfacente (Figura \ref{gradient_descent}a).
				 	\item Rallentare la ricerca del minimo a causa di un \textit{plateau} (Figura \ref{gradient_descent}b).
				 	\item Trovare un canyon "stretto" e continuare ad oscillare a causa dell'elevato contrasto tra i gradienti (Figura \ref{gradient_descent}c).
				 	\item Saltare un buon minimo a causa di una pendenza elevata (Figura \ref{gradient_descent}d).
				 \end{itemize} 
				 
			\subsection{Backpropagation}
				
				
	\chapter{Tecniche utilizzate}
		\section{Preprocessing}
			\subsection{Cross validation}
			\subsection{Grid search CV}
			\subsection{Scalatura dei dati}
		\section{Model Selection}			
			\subsection{Analisi delle componenti principali}
			\subsection{t-distributed Stochastic Neighbor Embedding}
		\section{Over-sampling}
			\subsection{Synthetic Minority Over-sampling Technique}
		
	\chapter{Esperimenti}
		\section{Dataset}
		\section{Esperimenti di classificazione}
		\section{Reingegnerizzazione dei dati}
		\section{Data augmentation}

	\chapter*{Conclusione}	
	
	\begin{thebibliography}{9}
		\bibitem{kriesel} David Kriesel, 2007, \textit{A Brief Introduction to Neural Networks}, available at \texttt{http://www.dkriesel.com}.

		\bibitem{shwartz} Shai Shalev-Shwartz, Shai Ben-David, 2014, \textit{Understanding Machine Learning: From Theory to Algorithms}, Cambridge University Press.

	\end{thebibliography}
	
\end{document}



